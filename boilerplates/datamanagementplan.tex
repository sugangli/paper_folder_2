% !TEX root = ../main.tex
\begin{center}
{\bf \large Data Management Plan}
\end{center}

\subsection*{Types of Data}
The data obtained during the proposed project will consist of (\textit{i}) performance and power measurements from the testbeds running benchmark applications; and (\textit{ii}) programmer feedbacks in the form of survey and interview from the user study of mobile application developers. 

\subsection*{Data and Metadata Standards} 
While there are no existing standards for the type of data that we are dealing with, we shall nevertheless define and adopt a well-documented format for all the data and metadata that we create for simulation and testbed.

\subsection*{Policies for Access and Sharing}
Summaries of the data generated by the experiments will be available in the form of plots and statistics in peer-reviewed publications. Upon request, we will freely make available the associated data used for generating the results within a reasonable period of time. We will implement a web-based data dissemination policy in accord with the rules and regulations of Rice University. There will be no charge for accessing this data.  However, to limit the load on our server, we may place data rate or time of access restrictions. We retain the right to use the data before opening it up to wider use but once we publish a paper we will release its corresponding data. Further, we will extensively make use of the arXiv e-print archive (\url{http://www.arxiv.org}) to post our research results. 

We do not anticipate that there will be any significant intellectual property issues involved with the acquisition of the data. In the event that discoveries or inventions are made in direct connection with this data, access to the data will be granted upon request once appropriate invention disclosures and/or provisional patent filings have been made. 
The data acquired and preserved in the context of this proposal will be further governed by policies pertaining to intellectual property, record retention, and data management from Rice University.

The PIs have a track record for sharing research data and tools with the research community. For example, PI Zhong has made open-source the K2 operating system at http://www.k2os.org in addition to many others that are available from \url{http://www.recg.org}. For another example, we have released the tools and data collected from our papers that appeared in MobiSys 2007 and MobiCom 2010 at CRAWDAD (the Community Resource for Archiving Wireless Data At Dartmouth, URL: \url{http://crawdad.cs.dartmouth.edu/}). Over 300 users have downloaded and used our data in their research according to CRAWDAD. PI Sarkar also maintains an active website for the Habanero project at \url{http://habanero.rice.edu}, as well as a download site (\url{https://wiki.rice.edu/confluence/display/PARPROG/HJDownload}) for the  latest releases of Habanero Java and DrHJ.

\subsection*{Provisions for Human Subject Protection}
The proposed project will use mobile application developers as human subjects in order to evaluate the usability of the proposed programming languages and design flow. The data collected from these human subjects include interview, survey and basic demographic information. Such data along with the design of interview, survey, and the entire user study will be made open-access after proper anonymization to remove any identity revealing information, according to the requirement of the Rice University IRB approval. 

PI Zhong has a track record of sharing human subject data with the community after anonymization. For another example, his group has anonymized the iPhone usage data of 34 users over six months and made it open-access online (\url{http://livelab.recg.rice.edu}).  


\subsection*{Policies and Provisions for Re-use, Re-distribution}

There will be no permission restriction placed on the data, which will be freely accessible, usable for any purpose, and redistributable. To disseminate experimental code for operating system, compiler, runtime and other tools, the PIs will use the open-source public repository at the Habanero project at \url{http://habanero.rice.edu} and a publicly accessible Wiki. The experiment code, designs and implementations of testbed will be open-sourced under a derivate of FreeBSD-like non-viral license when possible. 


\subsection*{Plans for Archiving and Preservation of Access}
The data will be preserved for at least three years beyond the award period, as required by NSF guidelines. The long-term strategy for maintaining, curating and archiving the data is via regular backup of data sets into suitable long term storage media such as optical or magnetic media stored in a secure location separate from the database server. There are no transformations necessary to prepare the data for preservation/data sharing. As far as metadata goes documentation will be preserved alongside the data in order to make the data reusable.

