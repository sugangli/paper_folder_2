% !TEX root = ../main.tex
%\pagebreak
\section{Education and Outreach Plan\label{sec:education}}
Our educational plan consists of three components, each tightly integrated with our research: (1) teaching embedded systems, wireless communication, and parallel and distributed systems, (2) engaging undergraduates and youth in research, and (3) encouraging female students in engineering.

The proposed project will contribute new curriculum content to the PIs' teaching in mobile computing, wireless communication, and parallel and distributed systems.  PI Zhang teaches senior elective courses ECE423 (Computer and Communication Networks) and ECE451 (Parallel and Distributed Computing) at Rutgers University. The proposed research will provide new lecture and laboratory modules for concepts in MIMO, radio tomography, vital sign sensing, and parallel applications. In particular, we plan to use diaster management as an application to drive communication protocol design and parallel computing in the two courses.  PI Zhong teaches a senior elective course ELEC/COMP424 Mobile \& Embedded System Design and Applications. The proposed research will provide new lecture and laboratory modules for concepts in distributed systems, hardware heterogeneity, and parallel programming. In particular, we plan to employ a face recognition application to teach about using specialized cores and synchronizing distributed states.


The proposed project will provide research opportunities to undergraduate students. PI Zhang has served as the faculty advisor for the Eta Kappa
Nu honor society for over 5 years. She has already worked with four undergraduate students from the honors program on related research projects, with publications coming out of several of these efforts (e.g.~\cite{jsspp03,sensorfusion05,xu:wise04,xu:mobihoc05}). PI Zhong has funded fifteen undergraduate students as research assistants in the past five years. Notably, he has four publications co-authored by undergraduates~\cite{rahmati2007mobilehci,liu2009percom,dong2009dac,dong2009islped} including two Best Paper awards~\cite{rahmati2007mobilehci,liu2009percom}. In this project, both teams plan to continue to work with undergraduate students. For each undergraduate assistant, we will define and tailor a project based on their interests, strengths, and career goals. Graduate students from both groups will work closely with undergraduates as collaborators and mentors. Moreover, we shall explore multiple funding opportunities for undergraduate research. Our track records include funding from the Brown Fund for Undergraduate Research, industry, and NSF REU supplements.

The proposed project will also provide a research platform to promote the awareness and interest in science and engineering among high-school students.  PI  Zhang has been engaged in an outreach to local high school students (Highland Park and Franklin High Schools in NJ). In this project, the Rutgers team will accept students from the Liberty Science Center's Partners in Science program and the WINLAB summer research program. PI Zhong will leverage his partnership with Harmony Science Academy, a local public school, to involve two students per semester in our research and mentor them for their Science Fair projects. PI Zhong has mentored five Harmony students in the past three years. One of them recently won the Second Place Award in Computer Science in the Texas Junior Academy of Science for her project under PI Zhong's guidance.


Ensuring that Science, Technology, Engineering and Mathematics (STEM) reaches as broad of a base as possible is an important activity that the investigators intend to focus on. PI Zhang has been actively involved in The Society of Women Engineers at Rutgers
University, where a series of workshops and luncheon meetings are
organized each semester, to give talks that emphasize the important
leadership roles open to women in EE/CS. She has supervised three female Ph.D. students, more than ten female master students and two female undergraduate students (both went to Cornell for graduate studies). One of her former Ph.D. students is now a tenured Associate Professor at University of South Carolina.
At CPS Week 2015, she served as one of the three panelists and discussed how to increase research impact for female researchers, which was very well received among female Ph.D. students and junior female faculty members. PI Zhong has also advised *** additional female PhD students. Moving forward, the PIs will continue their effort to encourage female students to engage in STEM.

\section{Broader Impact\label{sec:broader}}

The proposed project targets the following broader impact. We will leverage our collaborations with industry leaders in cyber-physical systems, including At\&T and Semens, to ensure a timely transfer of technologies and a broad impact on the commercial development of cyber physical systems. Our collaborations have already led to award-winning publications~\cite{liu2009percom,likamwa2011phonesense}, multiple patents, and likely product adoption.

We will publish research results and findings in academic conferences and journals and continue our tradition of demonstrating research results with prototypes at these conferences.
We plan to make our research hardware, software, and data open-source, as we have a strong record of doing this in the past. For example, PI Zhong have made the tools and data collected from~\cite{rahmati2007mobisys,liu2009percom,amirisani2010mobicom,shepard2010hotmetrics,wang2012www} open-source; PI Zhang has made the tools and data collected from~\cite{***} open-source.  More than 200 unique users have downloaded and used our data in their research according to CRAWDAD~\cite{crawdad}. Recently PI Zhong released the LiveLab data set, an extensive set of iPhone usage data from 34 users over six to 12 months~\cite{recgdownload}.
