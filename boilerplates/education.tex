% !TEX root = ../main.tex
%\pagebreak
\section{Education and Outreach Plan\label{sec:education}}
Our educational plan consists of three components, each tightly integrated with our research: teaching mobile system and parallel programming, engaging undergraduates in research, and outreach to high schools. 

The proposed project will contribute new curriculum content to the PIs' teaching in both mobile computing and parallel programming. PI Zhong teaches a senior elective course ELEC/COMP424 Mobile \& Embedded System Design and Applications. The proposed research will provide new lecture and laboratory modules for concepts in distributed systems, hardware heterogeneity, and parallel programming. In particular, we plan to employ a face recognition application to teach about using specialized cores and synchronizing distributed states.
PI Sarkar has recently created a sophomore-level course COMP322 Fundamentals of Parallel Programming required for all Computer Science majors at Rice. Habanero Java, a starting point of the proposed research,  is used in this course as the vehicle to teach parallel programming concepts. The proposed research will contribute lecture materials about new applications of parallel programming in mobile \& embedded computing as well as laboratory modules that use Habanero Java for programming smartphones. The two PIs plan to give guest lectures in each other's classes for this pedagogical cross-pollination. 


The proposed project will provide research opportunities to undergraduate students. We have funded fifteen undergraduate students as research assistants in the past five years. Notably, we have four publications co-authored by undergraduates~\cite{rahmati2007mobilehci,liu2009percom,dong2009dac,dong2009islped} including two Best Paper awards~\cite{rahmati2007mobilehci,liu2009percom}. For each undergraduate assistant, we will define and tailor a project based on their interests, strengths, and career goals. Graduate students from the PIs� groups work closely with undergraduates as collaborators and mentors. Moreover, we shall explore multiple funding opportunities for undergraduate research. Our track record includes funding from the Brown Fund for Undergraduate Research, industry, and NSF REU supplements.

The proposed project will also provide a research platform to promote the awareness and interest in science and engineering among high-school students. PI Zhong will leverage his partnership with Harmony Science Academy, a local public school, to involve two students per semester in our research and mentor them for their Science Fair projects. PI Zhong has mentored five Harmony students in the past three years. One of them recently won the Second Place Award in Computer Science in the Texas Junior Academy of Science for her project under PI Zhong's guidance.

\section{Broader Impact\label{sec:broader}}

The proposed project targets the following broader impact. We will leverage our collaborations with industry leaders in mobile computing, including Microsoft, Google, and Samsung, to ensure a timely transfer of technologies and a broad impact on the commercial development of mobile computing. Our collaborations have already led to award-winning publications~\cite{liu2009percom,likamwa2011phonesense}, multiple patents, and likely product adoption.

We will publish research results and findings in academic conferences and journals and continue our tradition of demonstrating research results with prototypes at these conferences. 
We plan to make our research hardware, software, and data open-source, as we have a strong record of doing this in the past. For example, PI Zhong have made the tools and data collected from~\cite{rahmati2007mobisys,liu2009percom,amirisani2010mobicom,shepard2010hotmetrics,wang2012www} open-source. More than 200 unique users have downloaded and used our data in their research according to CRAWDAD~\cite{crawdad}. Recently we released the LiveLab data set, an extensive set of iPhone usage data from 34 users over six to 12 months~\cite{recgdownload}. 
