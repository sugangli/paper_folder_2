% !TEX root = ../main.tex
\section{Results from Prior NSF Supports\label{sec:prior}}



\noindent \textbf{Yanyong Zhang} has, in the last 5 years, been the PI on NSF grants:
(i) CNS-0546072,  ``CAREER:PROSE: Providing Robustness in Systems of Embedded Sensors'', \$484K,  7/1/06--6/30/11;
(ii) CNS-0831186, ``CT - ISG: ROME: Robust Measurement in Sensor Networks'', \$400K, 09/01/08--08/31/11;
and co-PI on NSF grants
(iii) CNS-0910557, ``TC:Large: Collaborative Research: AUSTIN-- An Initiative to Assure Software Radios have Trusted Interactions'', \$410K, 9/1/09--8/31/12, (iv) CNS-1040735, ``FIA: Collaborative Research: MobilityFirst: A Robust and Trustworthy Mobility-Centric Architecture for the Future Internet'', \$2.73M, 9/1/10--8/31/13, and (v) CNS-1423020, ``NeTS: Small: Transmit Only: Cloud Enabled Green Communication for
Dense Wireless Systems'', \$498K, 9/1/14--8/31/17.

\emph{Intellectual Merit.} These research projects have led to several new contributions in the past 5 years to the areas of sensing and Internet of things~\cite{zan-etal:mdm10,sun2012association,sun2011improved,sun2012boomerang,moore2013building}, unobtrusive human context learning ~\cite{xu2012improving,xu2012towards,xu2013crowd++,xu2013scpl}, network virtualization~\cite{bhanage2011virtual,bhanage2011experimental}, and next generation Internet architecture design~\cite{vu2012dmap,sun2011improving,zhang2012content,liu2013secure,zhang2013using,li2013mobile,li2012popularity}.

\emph{Broader Impact}: These projects have had educational and outreach impacts. Dr. Zhang has advised 7 Ph.D. students,  has created Owl Platform (www.owlplatform.com), a low-power smart home sensing system based on results from prior NSF grants, has developed the global name resolution service prototype for the GENI national testbed, has redesigned the Computer Architecture, Database Systems, and Performance Evaluation courses at Rutgers-ECE. She is currently working with four Ph.D. students.



Both PIs Zhang and Zhong have received multiple NSF grants in the past five years. The most related to the proposed projects, one for each PI, are:

In a NeTS Small project, \emph{LAWN: Scaling Up Cellular Data Networks
  using a Large Number of Antennas}, CNS-1218700 (2012-15). {\bf Intellectual Merit:}~~PI Zhong
investigate
network architecture issues in applying massive MU-MIMO to cellular
networks. This project produced the Argos 64-antenna MU-MIMO base station~\cite{shepard2012mobicom} and provided the early
results that motivate some of the proposed research activities~\cite{shepard2013cellnet}. {\bf Broader Impact:} Argos was the largest many-antenna MU-MIMO base station prototype publicly known at its publication~\cite{shepard2012mobicom}. It influenced Samsung's development of its own 32-antenna MU-MIMO base station prototype~\cite{samsung2013fdmimo} and recent 3GPP's approval of a proposal to study massive MIMO for LTE~\cite{fdmimo}. %s 