% !TEX root = main.tex
\vspace{-10mm}
\section{Introduction\label{sec:intro}}
\
During natural or human-made disasters, the local infrastructure often becomes unavailable, and it is of paramount importance to develop efficient disaster management systems that can function without any infrastructure support. Central to such a system is the ability to rapidly detect whether there are live humans buried under piles of bricks and triage them. Further, these operations must be carried out in a efficient and cost-effective manner. 




\vspace{4pt}\noindent\textbf{Shortcomings with Existing Solutions:} Existing solutions are insufficient for the following reasons.

\vspace{4pt}\noindent\textbf{Tomography-Based Diaster Management with Many-Antenna Receivers:} In this proposal, we propose to design and develop a diaster management cyber-physical system (CPS) using tomography that is supported by many-antenna receivers.

\vspace{4pt}\noindent\textbf{Intellectual Merit:} In order to support a vision of conducting deep tomography-based disaster management using many-antenna receivers, the proposed effort presents a collaborative effort between a team at the Wireless Information Network Laboratory (WINLAB) at Rutgers University, and a team at Rice University to engage in a systems design and integration effort. The Rutgers team will lead the effort in protocol design and initial evaluation, while the Rice team will lead the effort in prototyping and testbed evaluation. We refer to the proposed tomography-based disaster management cyber-physical system as \emph{TomoMan} in this proposal. TomoMan has the following integral components: ***


As part of the broader impacts of the project, the team will reach out to *** 